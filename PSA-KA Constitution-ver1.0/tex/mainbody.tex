\chapter{Introduction}
Pakistan Students Association, Karlsruhe, Germany - is a self-administered community for Pakistani nationals to connect, integrate multi-cultural network, and collaborate on academic and non-academic projects or events. Open to all Pakistani nationals in Karlsruhe, as well as other individuals interested in Pakistani culture, festivals and academia. We share information on events, job opportunities, scholarships, and encourage members to share their research and experiences. Join us to build a strong and supportive community.

PSA-KA has been un-officially run through personal contacts and yahoo groups but in 2013 few Pakistani students created a Facebook page, named Pakistan Students Association, which helped coordinate and help each other. There was no official members of the association, but only run by some volunteers via this Facebook page. Now, it is the right time to structure the association and make everything documented.

Pakistan Students  Association Karlsruhe is managed by Pakistani students of Karlsruher Institute of Technology (KIT) and Hochschule Karlsruhe (HsKA), Germany. 
The objectives of PSA-KA are:
\begin{itemize}
	\item To help new students in their admissions at various KIT programs and help in initial settlement and get familiar to the German Culture.
	\item To propagate information about KIT master, bachelor, and doctoral programs offered in English/German to the Target audience and guide the interested people. 
	\item To promote and increase friendly and cordial relations between Pakistani, German, and other international students at KIT and other various universities of Karlsruhe.
	\item To facilitate and guide in communicating the difficulties and problems of Pakistani Students to the different administrative offices in specifically in KIT and generally in Karlsruhe.
	\item To arrange events, festivals, seminars and excursions activities for Pakistani Students of KIT.
	\item To collaborate, in academic and extra-curriculum activities,  with other student bodies of KIT or other universities of Karlsruhe and beyond.
	\item The association will not involve or propagate  any political or controversial events at the university campus, official pages or under any of its premises.
\end{itemize}

\chapter{Rules}
\section{Organizations’ Name}
Name of the organization shall be “PSA-KA: Pakistan Students Association, Karlsruhe” 
\section{Constitution Period Validity }
\begin{itemize}
	\item This constitution is valid for 2 years (from August 2023) and will be revised after expiration of this duration. 
	\item However, changes in current constitution require 2/3 majority of active members. 
\end{itemize}
\section{Objectives }
The objectives of PSA-KA are: 
\begin{itemize}
	\item To help and guide new students in their admissions and initial settlement. The initial settlement includes receiving new students, initial accommodation arrangements, city and other necessary registrations, assistance in enrollment in university, assistance in arranging health insurance and bank account activation. 
	\item PSA-KA will assign one person to each new comer, who will help in picking up, getting residence, insurance, enrollment, and introducing main super-markets and offices in the Karlsruhe.
	\item To promote and increase friendly and cordial relations between Pakistani, German, and other international students at Karlsruher Institute of Technology (KIT) and University of Applied Sciences (HsKA) Karlsruhe. 
	\item To guide them to interact with different services and office in Karlsruhe or in Germany/EU. 
	\item To arrange events, festivals, seminars and excursions for Pakistani Students/PSA-KA members. 
	\item To help, new job holders or families to karlsruhe, in finding the accommodation and integrating with the culture.
	\item To collaborate with Muslim Student Group, Karlsruhe (MSV) KA, Indians, Germans, KIT International Buddies etc. 
\end{itemize}
\section{Membership and Dues }
PSA-KA members are classified into two categories i.e.: 
\subsection{Active Members }
\begin{enumerate}
	\item Active membership is limited to all Pakistani students who are enrolled at KIT/HsKA (that have filled the PSA membership form). Foreign students cannot become active members and cannot hold any office. 
	\item A Non-Pakistani may also be awarded an honorary active membership on the recommendations of other active members and ultimately announced by the PSA.
	\item Active members are expected to participate in meetings and activities of PSA-KA and have right to vote and hold offices. 
	\item At present, volunteer-membership dues are 5-15 Euros per semester. These dues may be revised in future by PSA-Cabinet after taking approval from all active members. 
\begin{itemize}
	\item 	Bachelor Students: 5EUR
	\item 	Master Students: 10EUR
	\item 	PhD Students: 15EUR
\end{itemize}
\end{enumerate}

\subsection{Non-Active Members }
\begin{enumerate}
	\item Non-active membership shall include students’ families, KIT/HSKA faculty and staff.  
	\item Non-active members are not allowed to vote or hold offices. 
	\item Non-active members will pay 20 voluntarily, - Euros per year. 
	\item Foreigners can participate in PSA activities by paying per head cost of each event in which they are participating or as a guests depending on the event's nature. Event dues shall be decided in PSA-KA cabinet meeting. 
\end{enumerate}
\section{Founding Patrons  }
Following are the founding and permanent patrons of PSA, who has been involved during the registration process of PSA with KIT. Founding patrons will overlook the strategic affairs and management of PSA and will have the right to take corrective actions if necessary.
\begin{enumerate}
	\item Sajjad Hussain (KIT)
	\item Ibrahim Hameed (KIT)
	\item Anees Qumar Abbasi (KIT)
\end{enumerate}
Some of the important rules and duties for the founding patrons are:
\begin{itemize}
	\item The founding-patrons will remain the same. In case, if any addition of change in the founding-patrons is required, then the existing founding patrons and present board of directors can vote for a new inclusion.
	\item The founding-patrons has to handle the election process and voting mechanism in case when elections of new board-of-directors is required.
	\item The founding-patrons has to short-list the election candidates.
	\item The founding-patrons has to intervene and settle any disputes and misunderstandings among the board-of-directors and/or members.
\end{itemize}
\section{Board of Directors (Vorstandes)  }
\begin{itemize}
	\item The Board of Directors of the PSA-KA shall include three Spokesperson/Spokeswoman of the Association (Sprechers/Sprecherin der Vereinigung) HauptSprecher (Main Spokesperson), Other Spokespersons (General , and Finance Secretary). 
	\item The tenure of all cabinet members shall be one calendar year. 
	\item For all cabinet members, maximum duration for holding an office is 2 years. 
	\item All cabinet members must be active members of PSA-KA. 
	\item The number of cabinet-members or board-of-directors can be increased mutually by board-of-directors and the founding-patrons.
\end{itemize}
\section{Cabinet Responsibilities }
\subsection{HauptSprecher (Spokesperson) }
\begin{enumerate}
	\item Act as chair at PSA-KA general body meetings. 
	\item Supervise and co-ordinate activities of the organization. 
	\item Supervise the expenditure of all funds. 
	\item Be responsible for holding annual PSA-KA elections, events and excursions. 
	\item Appoint committee members when necessary. 
	\item Shall maintain the letter head of PSA-KA in his possession and could delegate it if needed. 
	\item Responsible for enforcement of PSA-KA constitution. 
	\item In case of dispute/draw Spokesperson shall have the final decision. 
	\item To interact with other organizations (like MSV, Asta etc.). 
	\item Liaison between Karlsruher Institute of Technology Karlsruhe and PSA-KA. 
	\item Official spokesperson of the association. xii. Coordinate with PSA-KA members. 
\end{enumerate}
\subsection{General Secretary }
\begin{enumerate}
	\item Act as the Spokesperson in absence of Spokesperson or in the interim period of his/her resignation. 
	\item Note, Compile and Keep minutes of meeting for each PSA-KA meeting. 
	\item Prepare the agenda and moderate each PSA-KA meeting. 
	\item Keep PSA-KA records. 
	\item Keep all the tangible assets of PSA-KA. 
	\item Responsible for planning and arrangement of activities. 
	\item Shall maintain active mailing lists of PSA-KA members. 
	\item Moderator/Administrator of social media pages like Facebook, yahoo groups, LinkedIn, WhatsApp, etc. 
	\item He shall be responsible for arranging extracurricular activities - trips, events, get-togethers etc. 
	\item Coordinate the initial settlements of new students throughout the year. 
	\item Document all the events i.e., resources utilized, event coverage - pictures, videos etc. 
	\item Maintain the official PSA-KA website, uploading credentials of new students, news, events etc. 
	\item Responsible to manage all the activities regarding information flow, communication and coordination among the members of PSA-KA.
\end{enumerate}
\subsection{Finance Secretary }
\begin{enumerate}
	\item Prepare budget for consideration of Cabinet. 
	\item Be responsible for all funds received and disbursed by the PSA-KA and for accounting of all bills, receipts and vouchers. 
	\item Collection of active membership dues. 
	\item Present in the general body meeting all the financial details of the previous year before handing over charge to the newly elected Finance Secretary. 
	\item Will be responsible for all monetary related interactions with international office in regards to all events. 
	\item Publish the summary of finance detail on quarterly basis. 
\end{enumerate}
\subsection{Members }
\begin{enumerate}
	\item To cooperate with cabinet members in carrying out different activities. 
	\item To promote and increase friendly and cordial relations among Pakistani, German and other students in Karlsruhe. 
	\item To abide by PSA-KA’s constitution. 
\end{enumerate}
\section{Elections of Cabinet/Board-of-Directors }
\begin{itemize}
	\item Elections for new term shall hold at the end of every tenure or when the founding-patrons decide it to do so. 
	\item The date and location for the Elections shall be announced and nominations for the cabinet positions shall be requested by the general secretary at least 2 weeks prior to the elections. 
	\item Each candidate has to present his agenda/layout for next tenure, and founding-patrons has to approve the nomination. This is to make sure that only serious nominations are entertained.
	\item Any active member of PSA-KA can nominate himself/herself for any cabinet position. 
	\item After the deadline for nominations, the General Secretary will announce the names of the potential candidates for each cabinet position, any candidate could step down from his/her position any time before the commencement of elections for that position. 
	\item Any of the current founding-patrons shall be the presiding officer for the elections. 
	\item The voting process shall be held through secret ballot. The cabinet members shall be elected by a simple majority vote of the members in attendance. In case of a tie, the Spokesperson shall cast the deciding vote. 
	\item The counting of ballots shall be done by the Spokesperson in presence of the respective candidates. 
	\item Each member shall be entitled to one vote only on all matters brought before the general body of the PSA-KA. 
	\item Any member can run for only one cabinet position at any given time. 
	\item Outgoing officers shall transfer power and any PSA-KA related material to newly elected PSA-KA Cabinet within two weeks. 
	\item If a cabinet member resigns from his/her post, bi-elections for remaining period will take place, if there remains more than three months in completion of cabinet’s tenure; otherwise, remaining members will distribute his/her responsibilities among themselves.  
\end{itemize}
\textbf{Exceptions: }

\begin{enumerate}
	\item When the office of the Spokesperson becomes vacant, the General Secretary shall become acting Spokesperson. 
	\item If the office of the Spokesperson and General Secretary become vacant simultaneously, the Finance Secretary shall become the acting Spokesperson. 
	\item The acting Spokesperson shall undertake all duties and powers of the Spokesperson until a new Spokesperson is elected. 
	\item The acting Spokesperson shall call a special or regular general body meeting for election as soon as possible but within one month. 
	\item When the office of General Secretary/Finance Secretary becomes vacant, the Spokesperson shall call a meeting for the election of new General Secretary/Finance Secretary as soon as possible but within one month. 
\end{enumerate}
\section{Impeachment of Cabinet Members }
\begin{itemize}
	\item Any active PSA-KA member can initiate impeachment proceedings against a cabinet member by presenting the charges in writing to the cabinet.  
	\item The cabinet upon receipt of the charges should initiate a poll for a General Body meeting. A simple majority in the poll would ensure the commencement of the General Body meeting.  
	\item Impeachment may be accomplished by a two-third vote of all active PSA-KA members present at the general body meeting.  
	\item Impeached member will be barred from holding any PSA-KA cabinet position in future. 
\end{itemize}
\section{Events’ advertisement }
Event details shall be maintained on PSA website. PSA-KA cabinet will also use social media platform like Facebook, google, linked-in, WhatsApp etc. for this purpose. 
\section{General Meetings }
\begin{itemize}
	\item The Spokesperson will announce the date, time and place of the general meeting by an email or any other means. 
	\item The agenda must be formulated and announced. 
	\item The agenda and announcement of the meeting must be done at least 2 weeks prior to the meeting. 
	\item At least one general body meeting must be held during a semester. 
	\item PSA-KA cabinet will ensure that at 1/3rd of total active members have given their consent to join the general body meeting; otherwise, the meetings will be called off. 
\end{itemize}
\section{Committees }
\begin{itemize}
	\item A Committee can be formed at any time to meet specific objectives defined by the Cabinet. 
	\item The head of the committee will be appointed by the cabinet. 
	\item The committee head will be presented with concrete objectives to fulfil or undertake. 
	\item The committee head is only responsible for presenting any actions the committee may take and the results of those actions to the Cabinet; whereas the members of the committee would be answerable only to the head of the committee. 
	\item The head of the committee in consultation with the cabinet will ask the PSA members to volunteer for membership in the committee. 
	\item In case there are not enough volunteers for the committee the cabinet can appoint members to fill in positions in the committee. 
	\item A committee shall be in session until its objectives have been completed, or suspended by the cabinet. 
	\item In case of any conflicts head of committee shall approach cabinet. 
\end{itemize}
\chapter{PSA-KA Constitution – version control sheet }
% Please add the following required packages to your document preamble:
% \usepackage[table,xcdraw]{xcolor}
% If you use beamer only pass "xcolor=table" option, i.e. \documentclass[xcolor=table]{beamer}
\begin{table}[h]
	\centering
	\begin{tabular}{|l|l|l|l|}
		\hline
		\rowcolor[HTML]{000000} 
		\multicolumn{1}{|c|}{\cellcolor[HTML]{000000}{\color[HTML]{FFFFFF} Version Number}} & \multicolumn{1}{c|}{\cellcolor[HTML]{000000}{\color[HTML]{FFFFFF} Published on (Date)}} & \multicolumn{1}{c|}{\cellcolor[HTML]{000000}{\color[HTML]{FFFFFF} Prepared by}} & \multicolumn{1}{c|}{\cellcolor[HTML]{000000}{\color[HTML]{FFFFFF} Valid Till}} \\ \hline
		\multicolumn{1}{|c|}{1.0}                                                           & \multicolumn{1}{c|}{August 2023}                                                        & \multicolumn{1}{c|}{Sajjad Hussain}                                             & \multicolumn{1}{c|}{15.06.2023}                                                \\ \hline
		&                                                                                         &                                                                                 &                                                                                \\ \hline
		&                                                                                         &                                                                                 &                                                                                \\ \hline
		&                                                                                         &                                                                                 &                                                                                \\ \hline
		&                                                                                         &                                                                                 &                                                                                \\ \hline
		&                                                                                         &                                                                                 &                                                                                \\ \hline
	\end{tabular}
\end{table}
